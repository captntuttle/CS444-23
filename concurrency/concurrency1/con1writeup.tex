\documentclass[10pt,letterpaper,onecolumn,draftclsnofoot]{IEEEtran}
\usepackage[margin=0.75in]{geometry}
\usepackage{listings}
\usepackage{color}
\usepackage{longtable}
\usepackage{tabu}
\usepackage{listings}
\definecolor{dkgreen}{rgb}{0,0.6,0}
\definecolor{gray}{rgb}{0.5,0.5,0.5}
\definecolor{mauve}{rgb}{0.58,0,0.82}
\definecolor{blue}{rgb}{0, 0, 1}

\lstset{frame=tb,
	language=bash,
	columns=flexible,
	numberstyle=\tiny\color{gray},
	keywordstyle=\color{blue},
	commentstyle=\color{dkgreen},
	stringstyle=\color{mauve},
	breaklines=true,
	breakatwhitespace=true,
	tabsize=4
}

\begin{document}

\begin{titlepage}

	\title{CS 444 - Fall 2017 - Concurrency 1 \\ Producer and Consumer}
	\author{Kristen Patterson, Kenneth Steinfeldt}
	\date{October 10, 2017}
	\maketitle
	\vspace{4cm}
	\begin{abstract}
		\noindent 
			\noindent
		Some sort of abstract.
	\end{abstract}
\end{titlepage}

\section{Command Logs}
The first thing we did to complete running QEMU was we created a repo and cloned in Yocto.

\begin{lstlisting}
git clone https://github.com/patterkr/CS444-23.git
cd CS444-23/
git clone git://git.yoctoproject.org/linux-yocto-3.19
\end{lstlisting}

Then we made sure to run the right version specified in the assignment.

\begin{lstlisting}
git checkout v3.19.2
\end{lstlisting}

Before we could run the qemu we had to copy over the starting kernel and the
drive file.

\begin{lstlisting}
cp /scratch/files/bzImage-qemux86.bin /scratch/fall2017/23/CS444-23/linux-yocto-3.19
cp /scracth/files/core-image-lsb-sdk-qemux86.ext4 /scratch/fall2017/23/CS444-23/linux-yocto-3.19
\end{lstlisting}

Then we entered bash to source the environment.

\begin{lstlisting}
source /scratch/opt/poky/1.8/environment-setup-i586-poky-linux
\end{lstlisting}

After it was properly sourced and the files were copied over, we ran the really
long qemu command replacing ???? with our port which was 5523.

\begin{lstlisting}
qemu-system-i386 -gdb tcp::5523 -S -nographic -kernel bzImage-qemux86.bin 
-drive file=core-image-lsb-sdk-qemux86.ext4,if=virtio -enable-kvm -net none 
-usb -localtime --no-reboot --append "root=/dev/vda rw console=ttyS0 debug".
\end{lstlisting}

Now that the QEMU was in debug mode, we opened up a new terminal and ran code
in bash to connect to it using our port 5523.

\begin{lstlisting}
source /scratch/opt/poky/1.8/environment-setup-i586-poky-linux
$GDB
target remote : 5523
continue
\end{lstlisting}

Next we needed to build Yocto so we opened another new terminal window and
copied over the configure files and made it using the allotted 4 jobs.

\begin{lstlisting}
cp /scratch/files/config-3.19.2-yocto-standard 
/scratch/fall2017/23/CS444-23/linux-yocto-3.19/.config
make -j4 all
\end{lstlisting}

Finally, to make sure it worked, we went back to the terminal where we
connected to the QEMU with GDB, logged into root, and ran the following commands.

\begin{lstlisting}
uname -r
\end{lstlisting}

Which showed that we were running in Yocto standard.

\section{Concurrency Solution}

\section{QEMU Flags Explanation}
\subsection{gdb}
Wait for gdb connection on tcp::5523.

\subsection{S}
Do not start CPU at startup.

\subsection{nographic}
Totally disable graphical output so QEMU is a simple command line application.

\subsection{kernel}
Use bzImage as kernel image.

\subsection{drive}
Defines a new drive file.

\subsection{enable-kvm}
Enables KVM full virtualization support. KVM is Kernel Virtual Machine.

\subsection{net}
Indicate that no network devices should be configured.

\subsection{usb}
Enable the usb driver.

\subsection{localtime}
Use local date and time.

\subsection{no-reboot}
Exit instead of rebooting

\subsection{append}
Use cmdline as kernel command line.

\section{Concurrency Questions}
\subsection{What do you think the main point of this assignment is?}

\subsection{How did you personally approach the problem?}

\subsection{How did you ensure your solution was correct?}

\subsection{What did you learn?}

\section{Version Control Log}

\begin{center}
	\begin{tabular}{|l|l|l|}
		\hline
		\textbf{Author} & \textbf{Date} & \textbf{Message} \\ \hline
		Kristen Patterson & October 4, 2017 & Initial commit \\ \hline
		Kristen Patterson & October 4, 2017 & Update README.md \\ \hline
		Kristen Patterson & October 4, 2017 & Built the kernel \\ \hline
		Kristen Patterson & October 7,2017 & Started concurrency programming, implemented buffer, need to research inline asm \\ \hline
		Ken Steinfeldt & October 8, 2017 & create stack(ish) and start p and c funcs \\ \hline
		Ken Steinfeldt & October 8, 2017 & begin produce and consume funcs \\ \hline
		Ken Steinfeldt & October 8, 2017 & Merge pull request 1 from patterkr/implement-stack \\ \hline
		Kristen Patterson & October 9, 2017 & Implemented rng and started tex doc \\ \hline
		Ken Steinfeldt & October 9, 2017 & filling in produce and consume \\ \hline
		Ken Steinfeldt & October 9, 2017 & attempted finish, not working \\ \hline
		Ken Steinfeldt & October 9, 2017 & trace statements, bug chasing \\ \hline
	\end{tabular}
\end{center}

\section{Work Log}
\subsection{Kristen Patterson}
I started working on Wednesday, October 4th. I first setup the environment and 
did all the commands the properly build and run the kernel. It took me about 
3 hours to setup the environment and build the kernel. Then on Saturday the 7th
I started the concurrency assignment by bringing in the mt19937ar.c and the 
Makefile. I also implemented some of the buffer to be used as well as a 
function to check if it was empty or full. Then on Monday the 9th I finished 
the random number generator function and finished my half of the tex document.

\subsection{Ken Steinfeldt}
I began working on Wednesday, October 4th to set up the environment and build
and run the kernel. Kristen, however, had already begun the process and had it
well in hand. On October 7 I began programming the concurrency portion of the 
assignment. On October 8 I submitted two commits. The first commit implemented 
an array based stack with push and pop functions this was about 3-4 hours of 
work. The second commit began work on the produce and consume functions. On 
October 9 I made 3 sequential commits. These commits filled out the produce, 
consume, and main functions to finish the program. This was about 5 hours of work.

\end{document}
