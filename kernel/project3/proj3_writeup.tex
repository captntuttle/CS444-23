\documentclass[10pt,letterpaper,onecolumn,draftclsnofoot]{IEEEtran}
\usepackage[margin=0.75in]{geometry}
\usepackage{listings}
\usepackage{color}
\usepackage{longtable}
\usepackage{tabu}
\usepackage{listings}
\definecolor{dkgreen}{rgb}{0,0.6,0}
\definecolor{gray}{rgb}{0.5,0.5,0.5}
\definecolor{mauve}{rgb}{0.58,0,0.82}
\definecolor{blue}{rgb}{0, 0, 1}

\lstset{frame=tb,
	language=bash,
	columns=flexible,
	numberstyle=\tiny\color{gray},
	keywordstyle=\color{blue},
	commentstyle=\color{dkgreen},
	stringstyle=\color{mauve},
	breaklines=true,
	breakatwhitespace=true,
	tabsize=4
}

\begin{document}

\begin{titlepage}

	\title{CS 444 - Fall 2017 - Project 3 \\ Encrypted Block Device}
	\author{Kristen Patterson, Kenneth Steinfeldt}
	\date{November 13, 2017}
	\maketitle
	\vspace{4cm}
	\begin{abstract}
		\noindent 
			\noindent
	\end{abstract}
\end{titlepage}

\section{Project Design}

\section{Project Questions}
\subsection{What do you think the main point of this assignment is?}

\subsection{How did you personally approach the problem?}

\subsection{How did you ensure your solution was correct?}

\subsection{What did you learn?}

\subsection{How should the TA test your program?}
Firstly start by taking the files from the patch file submitted and putting them in linux-yocto-3.19/drivers/block/.
Then after sourcing the environment correctly through the file provided in the opt/ directory do a git checkout for v3.19.2.
Afterwards type make -j4 all and wait for it to compile. After it is done use the following qemu command:

\begin{lstlisting}
qemu-system-i386 -gdb tcp::5523 -S -nographic -kernel arch/x86/boot/bzImage -drive file=core-image-lsb-sdk-qemux86.ext4,if=ide -enable-kvm -usb -localtime --no-reboot --append "root=/dev/hda rw console=ttyS0 debug"
\end{lstlisting}

After that is done open a new terminal and connect to it using the following code:

\begin{lstlisting}
$GDB
target remote : 5523
continue
\end{lstlisting}

After you are connected log in to root on the terminal with the kernel running.
When you log in you need to scp the .ko file over by doing the following.
I don't know how your directories are setup so I will show how I did mine.

\begin{lstlisting}
scp patterkr@os2.engr.oregonstate.edu:/scratch/fall2017/23/CS444-23/new_kernel/linux-yocto-3.19/drivers/block/sbd.ko /home/root
\end{lstlisting}

It should ask for a login password and then copy it over
Then you need to install the module while setting the key like so:

\begin{lstlisting}
insmod sbd.ko key="1234567890123456" keylen=16
\end{lstlisting}

After that is done then you need to run the command to make a partition and follow the steps to do so:

\begin{lstlisting}
fdisk /dev/sbd0
n //makes a new partition
p //selects primary
1 //selects partition number
//select first sector
//select last sector
p //builds partition
w //saves it
\end{lstlisting}

Next you need to make a filesystem for the partition

\begin{lstlisting}
mkfs.ext2 /dev/sbd0p1
\end{lstlisting}

When the filesystem is made you need to make a directory to mount it to:

\begin{lstlisting}
mkdir test
\end{lstlisting}

After the directory has been created you need to mount the filesystem:

\begin{lstlisting}
mount /dev/sbd0p1 test
\end{lstlisting}

After it has been mounted you can make a demofile that holds some text in it:

\begin{lstlisting}
touch test/demo
echo "Hello world!" > test/demo
\end{lstlisting}

Finally you can cat both the partition and the demo file to see the original contents and gibberish from the partition.

\begin{lstlisting}
cat test/demo
cat /dev/sbd0p1
\end{lstlisting}

This should prove we have successfully created an encrypted block device.

\section{Version Control Log}

\begin{center}
	\begin{tabular}{|l|l|l|}
		\hline
		\textbf{Author} & \textbf{Date} & \textbf{Message} \\ \hline
		Ken Steinfeldt & November 7, 2017 & creates project3 dir and start file \\ \hline
		Ken Steinfeldt & November 7, 2017 & assignment name header \\ \hline
		Kristen Patterson & November 8, 2017 & finished file \\ \hline
	\end{tabular}
\end{center}

\section{Work Log}
\subsection{Kristen Patterson}
I started work on Wednesday and worked all day until it was mostly complete. 
I started by finishing up the sbd.c file to use in the block device.
Then I found a working Makefile and Kconfig to transfer over with it. 
Afterwards I went about making sure it would compile and then figuring out how to run it to get it to work.

\subsection{Ken Steinfeldt}

\end{document}
