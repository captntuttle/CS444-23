\documentclass[10pt,letterpaper,onecolumn,draftclsnofoot]{IEEEtran}
\usepackage[margin=0.75in]{geometry}
\usepackage{listings}
\usepackage{color}
\usepackage{longtable}
\usepackage{tabu}
\usepackage{listings}
\definecolor{dkgreen}{rgb}{0,0.6,0}
\definecolor{gray}{rgb}{0.5,0.5,0.5}
\definecolor{mauve}{rgb}{0.58,0,0.82}
\definecolor{blue}{rgb}{0, 0, 1}

\lstset{frame=tb,
	language=bash,
	columns=flexible,
	numberstyle=\tiny\color{gray},
	keywordstyle=\color{blue},
	commentstyle=\color{dkgreen},
	stringstyle=\color{mauve},
	breaklines=true,
	breakatwhitespace=true,
	tabsize=4
}

\begin{document}

\begin{titlepage}

	\title{CS 444 - Fall 2017 - Project 2 \\ I/O Elevators}
	\author{Kristen Patterson, Kenneth Steinfeldt}
	\date{October 30, 2017}
	\maketitle
	\vspace{4cm}
	\begin{abstract}
		\noindent 
			\noindent
			The I/O scheduler exists in order to handle the efficient use of processor resources that way developers do not have to.
			An elevator algorithm and NOOP implementation are already provided.
			Using these we developed an SSTF implementation using the LOOK algorithm.
	\end{abstract}
\end{titlepage}

\section{Project Design}
Since we needed to implement our solution based off the current NOOP system, we had to take a look at that.
We also looked at LOOK which was required from the assignment and realized it fixed the starvation that SSTF would have.
The solution is based off of an elevator algorithm where the program goes up and down from top to bottom and back again through the requests.
The order of the requests is determined by the implemented LOOK algorithm.
The purpose of LOOK is to insert requests into the queue in such a way that minimizes seek time.
This is done by comparing a new request to the list head and placing the request either in front of it or behind it.

\section{Project Questions}
\subsection{What do you think the main point of this assignment is?}
We think the main point of this assignment was to further our knowledge about the kernel and specifically how the I/O scheduler works.
I also think it helped to increase our knowledge of the various algorithms that go into making it work, and using functions and tools that exist in the kernel.
This assignment also helped in that we finally modified the actual kernel rather than just compiling and running it.

\subsection{How did you personally approach the problem?}
Our approach was fairly simple.
First we brought in the implementation of NOOP to start with and also added in an elevator using elevator.h.
Then we changed it around in order to implement SSTF and use the LOOK algorithm.
Afterwards, we just compiled and ran the kernel with well placed print statements to show it working.

\subsection{How did you ensure your solution was correct?}
First in order to make sure our implementation was correct we focused on getting the kernel to actually build properly.
After it was able to build we added in different print statements to let us be able to see it actually running.
After we felt we had enough print statements showing our function running, we decided it was done.

\subsection{What did you learn?}
We learned how to manipulate the kernel a little better as well as manipulating the I/O scheduler.
We also learned various algorithms throughout the process of completing this assignment.

\subsection{How should the TA test your program?}
In order to test our program you must first setup the proper Linux Yocto.
This is done by doing the first part of assignment 1, namely, having it cloned from git, checking out the correct version copying over the required files for QEMU, and sourcing the right file.
After that is done then you must copy over our files sstf-iosched.c, Makefile, and Kconfig.iosched included in our code directory to the block directory located in the cloned linux-yocto.
After that is done you must manually change the I/O scheduler by typing make menuconfig and selecting enable block layer and then selecting I/O schedulers.
After that is done you just need to exit out of menuconfig and type make -j4 all.
After the kernel is made then you just need to input the following code.

\begin{lstlisting}
qemu-system-i386 -gdb tcp::5523 -S -nographic -kernel arch/x86/boot/bzImage -drive file=core-image-lsb-sdk-qemux86.ext3,if=ide -enable-kvm -net none -usb -localtime --no-reboot --append "root=/dev/hda rw console=ttyS0 debug"
\end{lstlisting}

\section{Version Control Log}

\begin{center}
	\begin{tabular}{|l|l|l|}
		\hline
		\textbf{Author} & \textbf{Date} & \textbf{Message} \\ \hline
		Kristen Patterson & October 24, 2017 & Started elevator implementation \\ \hline
		Ken Steinfeldt & October 28, 2017 & fills in add req \\ \hline
		Ken Steinfeldt & October 28, 2017 & fills in dispatch func \\ \hline
		Ken Steinfeldt & October 28, 2017 & add Kconfig file \\ \hline
		Kristen Patterson & October 28, 2017 & fixed slight error and added Makefile \\ \hline
		Ken Steinfeldt & October 29, 2017 & Kconfig file edit to include SSTF \\ \hline
		Ken Steinfeldt & October 29, 2017 & adds print statement for entry verification \\ \hline
		Kristen Patterson & October 30, 2017 & Add writeup and restructure directories \\ \hline
		Kristen Patterson & October 30, 2017 & Fixed some issues to try to get it to run \\ \hline
	\end{tabular}
\end{center}

\section{Work Log}
\subsection{Kristen Patterson}
I started work on Sunday the 24th by starting the implementation and finishing most of the elevator algorithm as well as incorporation of the NOOP implementation.
Then on Saturday I included the Makefile for the low-level kernel as well as fixed a slight error I saw in the program.
On Monday I did some restructuring of the directories and added the writeup for the assignment.

\subsection{Ken Steinfeldt}
I started working on Friday the 27th by studying the work that had already been completed and by researching how I would solve the remaining problems.
	On Saturday I filled in the remaining stubs and created a template Kconfig file.
	On Sunday I edited the config file and went over the code to make sure that I had not missed anything.

\end{document}
