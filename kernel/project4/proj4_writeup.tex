\documentclass[10pt,letterpaper,onecolumn,draftclsnofoot]{IEEEtran}
\usepackage[margin=0.75in]{geometry}
\usepackage{listings}
\usepackage{color}
\usepackage{longtable}
\usepackage{tabu}
\usepackage{listings}
\definecolor{dkgreen}{rgb}{0,0.6,0}
\definecolor{gray}{rgb}{0.5,0.5,0.5}
\definecolor{mauve}{rgb}{0.58,0,0.82}
\definecolor{blue}{rgb}{0, 0, 1}

\lstset{frame=tb,
	language=bash,
	columns=flexible,
	numberstyle=\tiny\color{gray},
	keywordstyle=\color{blue},
	commentstyle=\color{dkgreen},
	stringstyle=\color{mauve},
	breaklines=true,
	breakatwhitespace=true,
	tabsize=4
}

\begin{document}

\begin{titlepage}

	\title{CS 444 - Fall 2017 - Project 4 \\ SLOB SLAB}
	\author{Kristen Patterson, Kenneth Steinfeldt}
	\date{December 1, 2017}
	\maketitle
\end{titlepage}

\section{Project Design}
The slob already exists in linux but applies the first fit algorithm because it is better.
Knowing this, because we were told this in lecture, we simply modified the slob.c file that is included in the linux kernel by implementing the best fit algorithm.

The best fit algorithm works by simply looking through the slob for the closest fit to the required size.
When the best fit is found a new page is allocated with that memory.
This ensures the most space efficent use of memory.

\section{Project Questions}
\subsection{What do you think the main point of this assignment is?}
To understand that the linux kernel already contains logic that can be modified such as this slob slab.
We also believe that the use of system calls are a main point of this assignment.

\subsection{How did you personally approach the problem?}
Due to lecture we already knew that we only needed to modify the existing slob.c file found in the Linux kernel in order to complete the assignment.
Therefore, in order to implement the best fit algorithm into the slob we only needed to remove the first fit aspects and implement the best fit algorithms that we found online.

\subsection{How did you ensure your solution was correct?}
To be honest, We built the kernel successfully and booted it.
The boot process took significantly longer than previously and we counted that as success.
We ran out of time to properly test the implementation with code.

\subsection{What did you learn?}
We learned why best fit is not used, it takes a very long time to boot.
Furthermore we learned more about the slab and memory allocation in the Linux kernel as well as the use of systemcalls.

\subsection{How should the TA test your program?}

\section{Version Control Log}

\begin{center}
	\begin{tabular}{|l|l|l|}
		\hline
		\textbf{Author} & \textbf{Date} & \textbf{Message} \\ \hline
		Ken Steinfeldt & November 28, 2017 & initial commit \\ \hline
		Ken Steinfeldt & November 29, 2017 & Worked on best-fit slob \\ \hline
		Kristen Patterson & November 30, 2017 & finished best-fit slob \\ \hline
		Kristen Patterson & November 30, 2017 & started fist-fit slob \\ \hline
		Ken Steinfeldt & November 30, 2017 & finished most of first-fit \\ \hline
		Ken Steinfeldt & November 30, 2017 & add syscalls to table, 500 and 501 \\ \hline
		Ken Steinfeldt & December 1, 2017 & header files for sysslobfree() and sysslobused() \\ \hline
		Ken Steinfeldt & December 1, 2017 & fills in sysslobfree() and sysslobvoid() \\ \hline
	\end{tabular}
\end{center}

\section{Work Log}
\subsection{Kristen Patterson}
I started working November 30, after Ken had initially started the file.
I worked most of the day finishing up what he had started and when I was done I started on the first-fit slob.
I had finished most of the first-fit leaving the rest for Ken to sweep up and debug.

\subsection{Ken Steinfeldt}
I created the assignment and outline around the 28th.
I started most of the best-fit slob on the 29th.
On the 30th I came back in to finish the assignment and debugs.
Then on the first I added in all the supplementary testing files.
\end{document}
